\begin{frame}
\frametitle{How to build the tree structure? (II)}

Use the \textbf{cover tree} data structure
\begin{itemize}
%\item Build tree over arbitrary metric spaces
\item Good theoretical guarantees, fast to build
\pause
\item I invented the ``simplified cover tree'' \citep{izbicki2015faster}
\end{itemize}

%We use the \textbf{cover tree} data structure.

%\begin{block}{Assumption (informal)}
%There exists a distance metric $d$ over the classes that accurately represents
%\end{block}

\pause
\vspace{0.1in}
\textbf{Problem:}
\begin{itemize}
\item Tree should be built over optimal $\w_i^*$ to minimize $\lF{U^*}$/$\lF{V^*}$
\item We don't know $\w_i^*$
\end{itemize}

\pause
\vspace{0.1in}
\begin{block}{Assumption 3}
    \label{ass:metric}
    Let $\lambda \ge 1$, and $d$ be a distance metric over the labels such that for all labels $i$ and $j$,
\begin{equation}
    \frac 1 \lambda d(i,j)
    \le \ltwo{\star \w_i - \star \w_j}
    \le \lambda d(i, j).
\end{equation}
Let $c$ be the doubling dimension of this metric.
\end{block}

\end{frame}

%\begin{frame}
%\frametitle{Main Result!!!}

%%\begin{block}{Old Result}
    %%\begin{equation}
    %%\lF{\star W} \le \sqrt{dk}
    %%\end{equation}
%%\end{block}


%\begin{block}{Lemma 3 (informal)}
    %Under Assumptions 1-2 (highly technical) and 3, we have that
%%Let $c$ be the doubling dimension of the metric in Assumption 3.
%%Then,
    %%Let $c$ be the doubling dimension of the metric in Assumption 3.
    %%Then when $c\le1$, we have that
    %%\begin{equation}
        %%\lF{\star V} \le \lF{\star U} \le \tfrac{1}{\sqrt2}\lambda \sqrt{d\log_2 k} \le \lF{\star W} \le \sqrt{dk},
        %%\label{eq:c<=1}
    %%\end{equation}
    %%and when $c>1$, we have that
    %%\begin{equation}
        %%\lF{\star V} \le \lF{\star U} \le \sqrt{5}\lambda \sqrt{dk^{(1-1/c)}} \le \lF{\star W} \le \sqrt{dk}.
        %%\label{eq:c>1}
    %%\end{equation}
%\begin{equation}
%\lF{\star V} \le \lF{\star U} \le 
%O\left(
%\begin{cases}
%\lambda\sqrt{\frac{d \log k}{n}} & c \le 1\\
%\lambda\sqrt{\frac{d k^{1-1/c}}{n}} & c > 1
%\end{cases}
%\right)
%\le \lF{\star W} \le \sqrt{dk}
%\end{equation}
%\end{block}

%\vspace{0.1in}
%Lemmas 1 and 3 immediately imply

%\vspace{0.1in}
%\begin{block}{Theorem 1 (informal)}
%The tree loss satisfies
%\begin{equation}
%\text{generalization error} \le
%O\left(
%\begin{cases}
%\lambda\sqrt{\frac{d \log k}{n}} & c \le 1\\
%\lambda\sqrt{\frac{d k^{1-1/c}}{n}} & c > 1
%\end{cases}
%\right)
%\le
%O\left(\sqrt{\frac{dk}{n}}\right)
%\end{equation}
%\end{block}

%\end{frame}

%%%%%%%%%%%%%%%%%%%%%%%%%%%%%%%%%%%%%%%%%%%%%%%%%%%%%%%%%%%%%%%%%%%%%%%%%%%%%%%%

\begin{frame}
\frametitle{Main Result!!!}

%\begin{block}{Old Result}
    %\begin{equation}
    %\lF{\star W} \le \sqrt{dk}
    %\end{equation}
%\end{block}


\begin{block}{Lemma 3 (informal)}
    Under Assumptions 1-2 (highly technical) and 3, we have that
%Let $c$ be the doubling dimension of the metric in Assumption 3.
%Then,
    %Let $c$ be the doubling dimension of the metric in Assumption 3.
    %Then when $c\le1$, we have that
    %\begin{equation}
        %\lF{\star V} \le \lF{\star U} \le \tfrac{1}{\sqrt2}\lambda \sqrt{d\log_2 k} \le \lF{\star W} \le \sqrt{dk},
        %\label{eq:c<=1}
    %\end{equation}
    %and when $c>1$, we have that
    %\begin{equation}
        %\lF{\star V} \le \lF{\star U} \le \sqrt{5}\lambda \sqrt{dk^{(1-1/c)}} \le \lF{\star W} \le \sqrt{dk}.
        %\label{eq:c>1}
    %\end{equation}
\begin{equation}
\lF{\star V} \le \lF{\star U} \le 
O\left(
\begin{cases}
\sqrt{\frac{d \log k}{n}} & c \le 1\\
\sqrt{\frac{d k^{1-1/c}}{n}} & c > 1
\end{cases}
\right)
\le \lF{\star W} \le \sqrt{dk}
\end{equation}
\end{block}

\pause
\vspace{0.1in}
Lemmas 1 and 3 immediately imply

\vspace{0.1in}
\begin{block}{Theorem 1 (informal)}
The $U$ and $V$ parameterizations of the cross entropy loss when trained with SGD satisfy
\begin{equation}
\text{generalization error} \le
O\left(
\begin{cases}
\sqrt{\frac{d \log k}{n}} & c \le 1\\
\sqrt{\frac{d k^{1-1/c}}{n}} & c > 1
\end{cases}
\right)
\le
O\left(\sqrt{\frac{dk}{n}}\right)
\end{equation}
\end{block}

\end{frame}

%%%%%%%%%%%%%%%%%%%%%%%%%%%%%%%%%%%%%%%%%%%%%%%%%%%%%%%%%%%%%%%%%%%%%%%%%%%%%%%%

%\begin{frame}
%\frametitle{What's the doubling dimension $c$?}
%\end{frame}
